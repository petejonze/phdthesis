\chapter{Odds and ends}

\begin{chapabstract}
In this chapter we introduce some extra things that we couldn't fit in elsewhere. Experimenting with some of the things covered here will require going into the .cls source-file.
\end{chapabstract}

% ==============================================================================================
\section{The font}
The standard font is Bitstream-Charter. It is a bit strong/nicer than the standard modern. It has a fairly comprehensive set of mathmatical symbols, and can be scaled (\eg for use with dropcaps). The font can be changed by editing line in the .cls that reads: \verb|\renewcommand{\familydefault}{bch}|

% ==============================================================================================
\section{Filler text}
\lipsum[2]

% ==============================================================================================
\section{Comments}
It is \fxerror{sometimes}{somtimes} nice to be able too flag up errors for \fxerror{correction}{corection}. This can be done \fxerror{using}{useing} the \verb|fixme| package, \fxerror{which}{witch} allows you to write comments that appear in the margin. This paragraph contains a \fxerror{number}{numbr} of such comments. Can't see them? They are suppresed in final mode. Try running \verb|example_thesis_v1.tex| again, but this time using: \verb|\documentclass[isdraft, oneside, logo]{ihrthesis}|

% ==============================================================================================
\section{General tips}
Check the .log file for useful warnings, such as those concerning undefined references and citations.

% ==============================================================================================
\begin{chapacknowledgements}
This work was supported by the Medical Research Council, UK (Grant: U135097130).
\end{chapacknowledgements}

%% EOF